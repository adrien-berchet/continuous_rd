\documentclass[a4paper]{article}
\usepackage[T1]{fontenc}
\usepackage[utf8]{inputenc}
\usepackage{amsmath,amssymb}
\usepackage{amsfonts}
\usepackage{xcolor}
\usepackage[detect-all]{siunitx}
\usepackage{pgfplots}
\usepackage{natbib}
\bibliographystyle{abbrvnat}

\begin{document}

\title{Continuous reaction-diffusion process}
\author{Adrien Berchet}

\maketitle

\section*{Introduction}

The skin of oscellated lizards presents labyrinthine pattern that look close to the ones obtained by a quasi-hexagonal probabilistic cellular automaton. This goal of this work is to model the skin color dynamics during the lizards' life using a reaction-diffusion model \citep{Manukyan2017}. The skin of oscellated lizards contains 3 kinds of pigmentary elements: melanophores, xanthophores and iridophores, but the reaction-diffusion model is only applied to melanophores and xanthophores because the contribution of iridophores is neglictible in the process \citep{Nakamasu8429}.

\section{The continuous reaction-diffusion model}

The continuous reaction-diffusion model is represented by the following system of nonlinear partial differential equations \citep{Nakamasu8429}:

\begin{align}
	\dfrac{\partial u}{\partial t} & = F ( u , v , w ) - c_u u + D_u \nabla^2 u \\
	\dfrac{\partial v}{\partial t} & = G ( u , v , w ) - c_v v + D_v \nabla^2 v \\
	\dfrac{\partial w}{\partial t} & = H ( u , v , w ) - c_w w + D_w \nabla^2 w
	\label{rd_eq}
\end{align}
\begin{align}
	F ( u , v , w ) & = \begin{cases} \begin{aligned}
		0, && c_1 v + c_2 w + c_3 < 0 \\
		c_1 v + c_2 w + c_3, && 0 \leq c_1 v + c_2 w + c_3 \leq F_{max} \\
		F_{max}, && F_{max} < c_1 v + c_2 w + c_3
	\end{aligned} \end{cases} \\
	G ( u , v , w ) & = \begin{cases} \begin{aligned}
		0, && c_4 u + c_5 w + c_6 < 0 \\
		c_4 u + c_5 w + c_6, && 0 \leq c_4 u + c_5 w + c_6 \leq G_{max} \\
		G_{max}, && G_{max} < c_4 u + c_5 w + c_6
	\end{aligned} \end{cases} \\
	H ( u , v , w ) & = \begin{cases} \begin{aligned}
		0, && c_7 u + c_8 v + c_9 < 0 \\
		c_7 u + c_8 v + c_9, && 0 \leq c_7 u + c_8 v + c_9 \leq H_{max} \\
		H_{max}, && H_{max} < c_7 u + c_8 v + c_9
	\end{aligned} \end{cases}
	\label{rd_eq_interactions}
\end{align}

In equation \ref{rd_eq}, $u$ and $v$ components represent the densities of melanophores and xanthophores, respectively. The $w$ component represents a long-range factor (diffusion coefficient $D_w$ much larger than $D_u$ and $D_v$ ) produced by melanophores only.
In equation \ref{rd_eq_interactions}, $F$, $G$ and $H$ represents interactions among the chromatophores.

This model is implemented on a $2000 \times 380$ square lattice with step $\epsilon$ using periodic boundary conditions. In order to model the difference of skin thickness inside the skin scales versus their borders, regular hexagonal pattern (with hexagon side length $S$) is superimposed on the regular grid and the diffusion coefficients is reduced at the boundaries of hexagons: the Laplacian is multiplied by a factor $P \sin (\theta)$ at edges intersecting an hexagon boundary with angle $\theta$.

\section{Code implementation details}

Because of the quite large number of numerical elements, parallele computation is required in order to obtain the results in a decent time. The library Boost::ODEINT, which aims to solve ordinary differential equations, can be coupled with the Thrust library witch can use the GPU for computation.

\section{Validation}

\section{Results}

\bibliography{biblio.bib}

\listoftables
\listoffigures

\end{document}
